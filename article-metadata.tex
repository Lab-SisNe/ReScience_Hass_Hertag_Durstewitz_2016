% DO NOT EDIT - automatically generated from metadata.yaml

\def \codeURL{https://github.com/ReScience/ReScience-template}
\def \codeDOI{10.5281/zenodo.27944}
\def \dataURL{}
\def \dataDOI{}
\def \editorNAME{Tiziano Zito}
\def \editorORCID{None}
\def \reviewerINAME{Benoît Girard}
\def \reviewerIORCID{0000-0002-8117-7064}
\def \reviewerIINAME{Mehdi Khamassi}
\def \reviewerIIORCID{0000-0002-2515-1046}
\def \dateRECEIVED{09 June 2015}
\def \dateACCEPTED{12 August 2015}
\def \datePUBLISHED{14 August 2015}
\def \articleTITLE{ReScience Article Template}
\def \articleYEAR{2015}
\def \reviewURL{None}
\def \articleABSTRACT{This article is a proposition for a new article template for the ReScience C (computational replication) and ReScience X (experimental replication) journals. It is loosely based after Edward Tufte’s book style where the large left columns containes the main text and the right columns is used for auxiliary informations such as notes, captions or references. The template requires a standard TeXLive installation in order to compile it and this PDF has been compiled using TeXLive 2017 (pdflatex). Both the style, the layout and the colors of the template aim at giving ReScience a strong but subtle identity.}
\def \replicationBIB{M. Guthrie, A. Leblois, A. Garenne, and T. Boraud. Interaction between cognitive and motor cortico-basal ganglia loops during decision making: a computational study. In: Journal of Neurophysiology 109.12 (2013)}
\def \replicationDOI{10.1152/jn.00026.2013}
\def \contactNAME{Nicolas P. Rougier}
\def \contactEMAIL{Nicolas.Rougier@inria.fr}
\def \articleKEYWORDS{Latex, Template, ReScience}
\def \journalVOLUME{1}
\def \journalISSUE{1}
\def \articleNUMBER{1}
\def \articleDOI{None}
\def \authorsFULL{Nicolas P. Rougier and Nicolas P. Rougier}
\def \authorsABBRV{N.P. Rougier and N.P. Rougier}
\def \authorsSHORT{Rougier and Rougier}
\title{\articleTITLE}
\date{}
\author[1,2,3,\orcid{0000-0002-6972-589X}]{Nicolas P. Rougier}
\affil[1]{INRIA Bordeaux Sud-Ouest, Bordeaux, France}
\affil[2]{LaBRI, Université de Bordeaux, Institut Polytechnique de Bordeaux, Centre National de la Recherche Scientifique, UMR 5800, Talence, France}
\affil[3]{Institut des Maladies Neurodégénératives, Université  de Bordeaux, Centre National de la Recherche Scientifique, UMR 5293, Bordeaux, France}
