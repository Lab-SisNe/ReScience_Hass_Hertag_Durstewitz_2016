% DO NOT EDIT - automatically generated from metadata.yaml

\def \codeURL{https://github.com/ReScience/ReScience-template}
\def \codeDOI{}
\def \dataURL{}
\def \dataDOI{}
\def \editorNAME{}
\def \editorORCID{}
\def \reviewerINAME{}
\def \reviewerIORCID{}
\def \reviewerIINAME{}
\def \reviewerIIORCID{}
\def \dateRECEIVED{24 May 2018}
\def \dateACCEPTED{}
\def \datePUBLISHED{}
\def \articleTITLE{ReScience Article Template}
\def \articleDOMAIN{Editorial}
\def \articleYEAR{2018}
\def \reviewURL{None}
\def \articleABSTRACT{This article is a proposition for a new article template for both the ReScience C (computational replication) and ReScience X (experimental replication, upcoming) journals. It is loosely based after Edward Tufte’s book style where the large left column contains the main text and the right column is used for auxiliary informations such as notes, captions or references. The template requires a standard TeXLive installation in order to compile it and this PDF has been compiled using TeXLive 2017 (pdflatex). Both the style, the layout and the colors of the template aim at giving ReScience a strong but subtle identity.}
\def \replicationBIB{Visual Explanations: Images And Quantities, Evidence And Narrative, Edward R. Tufte, Graphics Press, 1997}
\def \replicationURL{https://www.edwardtufte.com/tufte/books_visex}
\def \replicationDOI{}
\def \contactNAME{Nicolas P. Rougier}
\def \contactEMAIL{Nicolas.Rougier@inria.fr}
\def \articleKEYWORDS{Latex, Template, ReScience}
\def \journalNAME{ReScience}
\def \journalVOLUME{None}
\def \journalISSUE{None}
\def \articleNUMBER{None}
\def \articleDOI{}
\def \authorsFULL{Nicolas P. Rougier}
\def \authorsABBRV{N.P. Rougier}
\def \authorsSHORT{Rougier}
\title{\articleTITLE}
\date{}
\author[1,2,3,\orcid{0000-0002-6972-589X}]{Nicolas P. Rougier}
\affil[1]{INRIA Bordeaux Sud-Ouest, Bordeaux, France}
\affil[2]{LaBRI, Université de Bordeaux, Institut Polytechnique de Bordeaux, Centre National de la Recherche Scientifique, UMR 5800, Talence, France}
\affil[3]{Institut des Maladies Neurodégénératives, Université  de Bordeaux, Centre National de la Recherche Scientifique, UMR 5293, Bordeaux, France}
